\documentclass[10pt]{examdesign} \usepackage{amsmath}
\newcommand{\blankspace}{$\underline{\phantom{something}}$}
\SectionFont{\large\sffamily} \Fullpages \ContinuousNumbering
\ShortKey \DefineAnswerWrapper{}{} \NumberOfVersions{5}

\class{{\Large Cell Quiz: Week 5}}

\begin{document}
\begin{shortanswer}[rearrange=yes]

\begin{block}
    \begin{question} What are the two sites within the cell at which
      protein synthesis is generally thought to occur?
      \begin{answer}
        ER, free ribosomes
      \end{answer}
      \examvspace{2cm}
    \end{question}
\end{block}

\begin{block}
    \begin{question} Where are misfolded secretory proteins eventually
      destroyed?
      \begin{answer}
        In the cytosol (cytoplasm).
      \end{answer}
      \examvspace{2cm}
    \end{question}

  \begin{question} In the absence of any signal or targeting sequence,
    what is the expected subcellular location for an arbitrary
    protein?
    \begin{answer}
      the cytosol
    \end{answer}
    \examvspace{2cm}
  \end{question}
\end{block}

\begin{block}
    \begin{question}
      As a secretory protein is transported through the RER membrane,
      a block of sugars is transferred from a lipid molecule embedded
      in the membrane to an amino acid on the protein.  The block of
      sugars is identified as an N-linked oligosaccharide.  Which
      amino acid on the secretory protein is most likely to serve as
      the acceptor for the oligosaccharide?
      \begin{answer}
        An asparagine residue.
      \end{answer}
      \examvspace{2cm}
    \end{question}

    \begin{question} How and where is the asymmetry of the
      phospholipid bilayers initially established?
      \begin{answer}
        In the ER during lipid and protein synthesis.
      \end{answer}
      \examvspace{2cm}
    \end{question}
\end{block}

\begin{block}
  \begin{question}
    You take a DNA sequence that codes for a normal NLS and fuse it to
    the gene coding for a non-nuclear protein called ovalbumin and
    inject it into the cytoplasm of the cell where the new fusion
    protein (ovalbumin) is manufactured.  You then monitor what
    happens to the protein.  What happens?
    \begin{answer}
      The fusion protein is transported to the nucleus.
    \end{answer}
    \examvspace{2cm}
  \end{question}
\end{shortanswer}

\begin{fillin}
  \begin{question}
    A transport receptor that moves proteins from the
    \word{{cytoplasm}{nucleus}} to the \word{{nucleus}{cytoplasm}} is
    called a \blank{\word{{importin}{exportin}}}
  \end{question}
\end{fillin}

\end{block}
\begin{block}
  \begin{multiplechoice}[rearrange=yes]
    \begin{question}
      Nuclear lamina disassembly prior to mitosis is thought to be
      induced by \blankspace of \blankspace by a specific \blankspace.

      \choice[!]{phosphorylation, lamins, protein kinase}
      \choice{phosphorylation, lamins, protein phosphatase}
      \choice{acetylation, laminins, protein kinase}
      \choice{phosphorylation, laminins, protein kinase}
      \choice{acetylation, lamins, protein acetylases}
    \end{question}
    \begin{question}
      You have isolated the gene for a protein, nuculin (a protein
      that George W. Bush believes resides in the nuculus) that is
      normally localized to the nucleus.  You locate in the gene the
      portion of the coding sequence that codes for the nuclear
      localization signal (NLS) and engineer it so that a positively
      charged amino acid in the signal is replaced by a nonpolar amino
      acid.  You then insert the gene into some cells and observe
      where nuculin goes after its synthesis.  What is the effect, if
      any?

      \choice{There is no effect.}
      \choice{Nuculin moves to the nucleus as it usually does.}
      \choice[!]{ Nuculin is not moved to the nucleus.}
      \choice{ Nuculin is not made by the cell at all.}
      \choice{ Muculin}
    \end{question}
  \end{block}
\end{multiplechoice}
 \end{document}
